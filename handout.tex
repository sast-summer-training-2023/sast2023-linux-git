\documentclass{article}
\usepackage{extramarks, fancyhdr, minted}
\usepackage{main}

\pagestyle{fancy}
\lhead{\nouppercase{\lastleftmark}}
\rhead{\nouppercase{\lastrightmark}}

\def\refname{References}

\title{SAST\,2023 Summer Training\\CTF | A Warmup for Linux \& Git}
\author{jkjkmxmx @ yhx-12243}
\setdate{2023}{7}{9}

\begin{document}
	\maketitle
	\let\labelitemi\labelitemiii

	\section{Linux}

	\subsection{Introduction}

	\subsubsection{什么是 Linux}

	Linux 是一种自由和开放源码的类 UNIX 操作系统,其内核由 Linus Torvalds 在 1991 年发布。

	Linux 也是自由软件和开放源代码软件发展中最著名的例子,只要遵循 GNU,任何个人和机构都可以自由地使用 Linux 的所有底层源代码,这使得它得到来自全世界软件爱好者和组织的开发支持。

	Linux 以各种形式被广泛应用在各个领域,包括但不限于:
	\begin{itemize}
		\itemsep0pt
		\item 服务器、主机、超级计算机;
		\item 嵌入式系统(机顶盒、移动设备等);
		\item 基础设施(红绿灯、工业传感器)。\cite{net9-linux}
	\end{itemize}

	\subsubsection{发行版}

	我们平时使用的 ``Linux'' 严格来说是 Linux 发行版本,而 Linux 狭义上单指操作系统的内核。

	发行版本在内核的基础上还包括安装工具、系统配置、图形桌面界面、各种 GNU 软件等,使得这个系统能够适用于各种使用目的。

	常见的 Linux 发行版有 Debian、Ubuntu、Fedora、CentOS、Arch Linux 等。不同的发行版使用的软件不同,有些发行版本是设计成专门的目的(比如 Kali 用于网安)。\cite{net9-linux}

	以 Ubuntu 为例,Ubuntu 在每个偶数年的 4 月 (每两年) 发布一个长期支持版 (\textcolor{fuchsia}Long-\textcolor{fuchsia}Term \textcolor{fuchsia}Support),每 6 个月发布一个版本。

	我们提供的服务器安装的发行版是 Ubuntu 22.04 (LTS),代号 Jammy Jellyfish。

	\subsubsection{接触 Linux 的方式}

	我们接触 Linux 的方式通常有以下几种:\cite{net9-linux}

	\begin{itemize}
		\itemsep0pt
		\item 安装 Linux 系统 (单系统、Windows \& Linux 双系统);
		\item WSL (英语:Windows Subsystem for Linux);
		\item 虚拟机 (Virtualbox、Docker);
		\item 服务器远程连接 (SSH、RDP);
	\end{itemize}

	有条件的同学可以给电脑装一个 Linux 系统 (或双系统),\sout{这样你就能在各种配置中度过一段漫长而有趣的时光了}。

	对于其他同学,Windows 的话推荐使用 \href{https://learn.microsoft.com/zh-cn/windows/wsl/install}{WSL\,2} (毕竟是 Type-1 的虚拟机,有性能支持),\sout{如果只想要一个终端的话}\linebreak[1]\sout{可以用 Git Bash};对 Mac\,OS 的话原生的终端就有比较接近的体验 (推荐使用 \href{https://iterm2.com/}{iTerm\,2}),或者使用 Docker。当然有服务器的同学可以直接用各种服务器 (现代的服务器基本都是 Linux)。

	\subsection{课前准备}

	参考链接:
	\begin{itemize}
		\itemsep0pt
		\item \url{https://learn.microsoft.com/zh-cn/windows/wsl/install}
		\item \url{https://blog.csdn.net/yushuzhen2008/article/details/104944579}
	\end{itemize}

	\subsection{GUI, CLI, Shell 和 Terminal}

	见 \url{https://docs.net9.org/basic/linux/#basic-concepts}。

	\subsection{学习资料}

	网上的 Linux 学习资料非常多,同学们需要善于寻找资源:

	\begin{itemize}
		\itemsep0pt
		\item \url{https://docs.net9.org/basic/linux/}\quad\textcolor{red}{$\gets$ \textbf{务必熟读,其中的内容本文不再重复出现}}
		\item \url{https://www.runoob.com/linux/linux-tutorial.html}
		\item \url{https://ryanstutorials.net/linuxtutorial/}
		\item \url{https://missing.csail.mit.edu/}
	\end{itemize}

	当遇到问题时,要善于利用 Bing、Google 等搜索引擎,以及 \href{https://stackoverflow.com/}{Stack Overflow}、\href{https://stackexchange.com/}{Stack Exchange} 等。当然在 2023 年的今天,ChatGPT/GPT-4 等工具也一直是你的好伙伴 (对于比较基础的问题效果还是很不错的)。

	当然,在遇到不会的命令时,还要养成看文档的习惯,这一部分可以参考 \hyperref[sss:linuxdoc]{\ref*{sss:linuxdoc} 节}。

	\subsection{Shell}

	Shell 是终端编程中常见的 ``语言'',Shell 变成为典型的命令式编程 (区别于声明性编程),常用于写快餐式的脚本以及充当 ``胶水'' 的作用,\sout{和 Python 比较类似}。

	\subsubsection{Shell 方言}

	Shell 并不是一种固定的语言,而是衍生是许多的方言 (扩展),如 \texttt{sh} (原生), \texttt{bash}, \texttt{zsh}, \textcolor{gray}{\texttt{dash}, \sout{\texttt{fish}, \texttt{csh}, \texttt{ksh}}} 等。

	比较常见的是 Bash (Ubuntu 默认) 和 Z shell (zsh, Mac\,OS 默认)。当然这些 shell 自身也会有相对应的扩展,比如服务器上默认安装的 Z shell 扩展 \href{https://ohmyz.sh/}{``Oh My Zsh''}。

	通过 \texttt{type} 命令\footnote{有些教程用建议使用 \texttt{which} 命令,但是在有一些 Shell (Sh 和 Bash) 中 \texttt{which} 是\Emm{外部命令},因此无法查阅如 \texttt{if} 等\Emm{内部命令}。而 \texttt{type} 在常见的 Sh, Bash, Z shell 中均为内部命令,可以查询到所有命令的信息。}可见\textcolor{red}{形如 \texttt{cd}, \texttt{if}, \texttt{pwd}, \texttt{alias}, \texttt{setopt}/\texttt{shopt} 等命令返回了 ``shell 内建''/``shell 关键字'',它们统称\Emm{内部命令};而形如 \texttt{ls}, \texttt{wc}, \texttt{ps}, \texttt{cp}, \texttt{find}, \texttt{mkdir}, \texttt{sudo} 等命令会返回一个对应文件所在的路径,它们统称\Emm{外部命令}}\footnote{实际上还有别名,函数等形式,可以使用 \texttt{type -w} (Bash 中为 \texttt{type -t}) 来查看类别。}。% type -w ls cd wc tc if none

	\begin{itemize}
		\item \Emm{内部命令}是嵌入在各个 Shell 中的 (写在对应的源码里),\sout{就像你们写 Shell 大作业时的一堆 if/else 里讨论的命令},\textcolor{green!80!black}{不创建新进程,执行速度较快},缺点是\textcolor{red}{不同的 Shell 的实现可能会不一样,就是说方言可能会比较多}。
		\item \Emm{外部命令}是 Linux 系统中独立存在的外部程序,\textcolor{green!80!black}{由于是外部程序,因此在所有 Shell 看来表现几乎一样},\textcolor{red}{调用时会创建新的子进程}。

			当输入文件名为一个标识符时,Shell 通常不会解析为相对路径,而是解析为命令,然后按照内部优先,外部其次的顺序进行查找。其中外部命令的查找会根据 \texttt{PATH} 环境变量\textit{按顺序依次查找}。

			具体的优先级可以使用 \texttt{type -a <command>} 来查看,以得知有哪些结果被覆盖。
	\end{itemize}

	\subsection{在 Linux 中查文档}
	\label{sss:linuxdoc}

	在 Linux 以及之后的各种工作中,查文档是必不可少的技能。除上网查询外,Shell 中查文档的方法主要有如下几种:
	\begin{itemize}
		\itemsep0pt
		\item \texttt{<command> --help},\textbf{只能用于 (文档良好的) 外部命令} (大多数外部命令都有良好的文档)。
		\item \texttt{help <command>},\textbf{Bash 中查询内部命令的方式}。
		\color{red}\item \texttt{man <command>},比较通用,\textbf{非常牛逼,非常详细,而且通常由官方撰写的命令手册},强烈推荐使用!

			\normalcolor\texttt{man man} 可以查询 \texttt{man} 的使用方式,\sout{没想到吧,这就叫元学习!}

			所有这些手册 (\texttt{man}ual) 分为 9 类\cite{man-man}:
			\begin{enumerate}
				\item 可执行程序或 shell 命令;
				\item 系统调用 (内核提供的函数);
				\item 库调用 (程序库中的函数);
				\item 特殊文件 (通常位于 \texttt{/dev});
				\item 文件格式和规范,如 \texttt{/etc/passwd};
				\item 游戏;
				\item 杂项 (包括宏包和规范),如 \texttt{man(7)}, \texttt{groff(7)}, \texttt{man-pages(7)} 等;
				\item 系统管理命令 (通常只针对 root 用户);
				\item 内核例程 (非标准)。
			\end{enumerate}

			通过 \texttt{man <\kern0pt类别\kern0pt> <\kern0pt命令\kern0pt>} 可以具体指定在哪一类中查。大家可以体会一下 \texttt{man 1 printf} 和 \texttt{man 3 printf} 的差别\footnote{\url{https://www.cnblogs.com/kelamoyujuzhen/p/9822224.html}, \url{https://blog.csdn.net/weixin_50502862/article/details/125583685}}。

			另外一个常用的选项是 \texttt{-k},就当你不知道要对什么 \texttt{man} 时,查询所有能匹配正则表达式的手册列表。

			此外,进入手册后,Shell 通常会使用一个叫做 \texttt{less} 的 Pager (说人话就是你和使用 \texttt{less} 命令一样),这是一个和 Vim 类似的环境。此时你可以 \texttt{man less} 或直接按下 ``\texttt h'' 键 (或直接搜索) 寻求帮助,比如常见的 ``\texttt /'' 表示搜索。

			有关 \texttt{man} 的更多用法详见 \texttt{man man}。
		\color{fuchsia}\item\phantomsection\label{item:docbuiltin}\texttt{man bash}/\texttt{man builtins} (Bash) 和 \texttt{man zshbuiltins} (Z shell) 用于查询所有内部命令的文档。

			\normalcolor\sout{相信大家也被 Z shell 里没有 \texttt{help} 命令而且 \texttt{man if} 跳出的东西云里雾里所困扰吧(逃}
		\item \texttt{tldr <command>}。对于一些不常用的命令,你不打算花时间精雕细琢地学习,只想快速知道最常见的用法 (范式,模板)。这时,\texttt{tldr} 是一个非常不错的选择。\texttt{tldr} (Too long; didn't read) 提供了一些命令的非常常见的套路用法,以加快你的工作效率。
	\end{itemize}

	{\color{red}\bfseries 注:在这之后提到如果有没出现的 ``前置技能'',就表示建议使用 \texttt{man} 或 \texttt{tldr} 工具进行相关自学。\par
	再次提醒,学会主动查手册是一项非常非常非常重要的技能,之后的命令介绍会非常简略,希望大家学会自己查 \texttt{tldr} 获取主要用法,查 \texttt{man} 获取手册!!!\par
	再次提醒,学会主动查手册是一项非常非常非常重要的技能,之后的命令介绍会非常简略,希望大家学会自己查 \texttt{tldr} 获取主要用法,查 \texttt{man} 获取手册!!!\par
	再次提醒,学会主动查手册是一项非常非常非常重要的技能,之后的命令介绍会非常简略,希望大家学会自己查 \texttt{tldr} 获取主要用法,查 \texttt{man} 获取手册!!!}

	\subsection{SSH 相关}

	\textcolor{fuchsia}{[\textit{主要命令:\texttt{ssh}, \texttt{scp}, \texttt{ssh-keygen}, \texttt{ssh-copy-id}}]}

	命令的用法请使用 \texttt{man} 以及 \url{https://docs.net9.org/basic/linux/#ssh} 进行学习。值得注意的一点是,可以在家目录下配置 \texttt{.ssh/config} 文件来将记住常用的服务器配置 (VSCode 也会使用这个!) 详细的规范见 \verb!man ssh_config!。

	\subsection{Shell 语法}

	\textcolor{fuchsia}{[\textit{方便起见,本节内容偏向 Z Shell。}]}

	见\autoref{table:grammar},扩展自 \cite{lbwvssl}。官方文档可参考 \url{https://zsh.sourceforge.io/Doc/Release/index.html}。

	\begin{table}[htb]
		\centering
		\begin{tabular}{|c|c|c|c|}
			\hline
				命令 & 例子 & 介绍 & 手册 \\
			\hline\hline
				\texttt{\# comment} & \texttt{\# this is a comment} & 注释 & \texttt{man zshmisc} \\
			\hline
				\texttt{<VARIABLE>=<VALUE>} & \vbox to18pt{}\vtop to12pt{}\texttt{ANSWER=42} & \parbox{140pt}{定义一个变量,\textbf{\color{red}注意等号两边不能有空格,Shell 大小写敏感}} & \texttt{man zshparam} \\
			\hline
				\verb!${VARIABLE}! & \verb!echo ${ANSWER}! & 使用变量,无歧义时可省略大括号 & \texttt{man zshparam} \\
			\hline
				\verb!$NUM! & \vbox to18pt{}\vtop to12pt{}\verb!echo $1! & \parbox{140pt}{使用参数,如 \texttt{./script arg1 arg2 arg3} 中 \texttt{\$2} 即为 \texttt{arg2}} & \texttt{man zshparam} \\
			\hline
				\verb!$@! 或 \verb!$*! & \verb!echo $*! & 所有参数的数组 & \texttt{man zshparam}\footnotemark \\
			\hline
				\texttt{expr} & \texttt{expr 2 + 4} & 计算数学表达式 & \texttt{man expr} \\
			\hline
				\verb!`COMMAND`! 或 \verb!$(COMMAND)! & \verb!echo $(expr 2 + 4)! & 取命令结果 & \texttt{man zshexpn} \\
			\hline\hline
				\verb!{A,B,C}! & \verb!cat file{A,B,C}.txt! & 按照分配律展开 & \texttt{man zshexpn} \\
			\hline
				\verb!{low..high}! & \verb!touch test{1..10}.in! & 范围展开 & \texttt{man zshexpn}  \\
			\hline
				\verb!~! & \verb!cd ~! & 展开你的家目录 & \texttt{man zshexpn} \\
			\hline
				\verb!~USER! & \verb!echo ~root! & 展开某个用户的家目录 & \texttt{man zshexpn} \\
			\hline
				\texttt ?, \texttt *, \texttt{**}, \texttt{[...]} & \verb!rm **/A?[1-5].log! & 文件名匹配 (\href{https://en.wikipedia.org/wiki/Glob_(programming)}{Glob}) 语法 & \texttt{man zshexpn} \\
			\hline
		\end{tabular}
		\caption{常用语法}
		\label{table:grammar}
	\end{table}

	其它复杂的语法 (如 \verb!() (()) [] [[]] {} if for case select! 定义函数等) 可见 \texttt{man zshexpn},\texttt{man zshparam},\texttt{man zshmisc} 等。所有文档可通过 \verb!ls /usr/share/man/man1/zsh*! 或 \texttt{man zshall} 查询,当然也可查询 \url{https://zsh.sourceforge.io/Doc/Release/index.html} (其实内容是一样的说) 或网上相关教程 (不一定靠谱)。

	\subsubsection{环境变量}

	环境变量是一类特殊的变量,有点像局部变量,它的一大特性是\textbf{可以继承给它的子进程},从而广泛应用于外部命令们的参数配置。

	环境变量的定义有如下几种方式:
	\begin{itemize}
		\itemsep0pt
		\item \texttt{export <VARIABLE>=<VALUE>}:定义环境变量。
		\item \texttt{<VARIABLE>=<VALUE> <command>}:在这一条命令中拥有该环境变量,命令结束后变量消失 (\texttt{man zshmisc})。
		\item 使用其它程序设置。
	\end{itemize}

	环境变量可以用程序或 \texttt{unset <VARIABLE>} 清除。

	\begin{itemize}
		\itemsep0pt
		\item \texttt{declare} / \texttt{typeset} 可用于查看所有定义的变量。
		\item \texttt{env} 可用于查看所有定义的环境变量,以及运行相关程序。
	\end{itemize}

	\subsubsection[调用其它脚本]{调用其它脚本\protect\cite{lbwvssl}}

	在 Shell 中调用其它脚本有两种方式:\texttt{source} (简写 \texttt .) 或子 Shell (\texttt{zsh <script>} 或使用 \href{https://en.wikipedia.org/wiki/Shebang_(Unix)}{Shebang})。

	主要区别是:\texttt{source} 方法在本 shell 执行,子 Shell 方法会新建子终端执行。

	脚本中的局部变量定义仅属于执行它的终端

	\begin{minted}[frame=single,linenos=true,rulecolor=blue,escapeinside=||]{shell}
echo 'ANSWER=42' > script.sh
zsh script.sh # 变量在子终端定义
echo ${ANSWER} # 对本终端,变量不可见
source script.sh # 变量在本终端定义,也可用 . script.sh
echo ${ANSWER} # 变量可见
	\end{minted}

	\subsection{Shell 选项 (可选)}

	\textcolor{fuchsia}{[\textit{方便起见,本节内容偏向 Z Shell。}]}

	Shell 选项有些地方有称之为 Shell 专有变量。

	\subsection{一些命令}

	这里还有一些\sout{可能在 Puzzle 中用得到的}命令,具体方法请自行使用 \texttt{man} 和 \texttt{tldr} (记住,他们真的是很好的老师,我已经没什么好写的了)。
	\footnotetext{不同 Shell 对这种变量的处理方式大相径庭。Bash 的处理比 Z Shell 更加混沌邪恶。在 Z Shell 中 \texttt{\$*} = \texttt{\$argv},以及对应的 \texttt{\$ARGC},具体区别可见 \texttt{man zshparam}。}

	\subsubsection{进程}

	\begin{itemize}
		\itemsep0pt
		\item \texttt{ps} / \texttt{top} / \texttt{htop}:查询进程信息
		\item \texttt{uptime}:查询系统启动时间,用户数以及平均负载 \sout{(其实就是 \texttt{top} 的第一行)}
		\item \texttt{kill} / \texttt{killall}:杀死进程
		\item \texttt{... \&} / \texttt{fg} / \texttt{bg} / \texttt{jobs} / \texttt{nohup}:前后台切换,\sout{避免《人类的群星闪耀时》}(查不到?看看\hyperref[item:docbuiltin]{这里}!)
	\end{itemize}

	\subsubsection{文件系统}

	\begin{itemize}
		\itemsep0pt
		\item \texttt{cat}:输出并连接文件
		\item \texttt{tee}:将输入复制一份 (标准输出和文件)
		\item \texttt{split}:分裂文件 (用于制作分卷压缩包)
		\item \texttt{dd}:拷贝文件的底层接口
		\item \sout{\texttt{tac} / \texttt{rev}:花里胡哨地输出文件}
		\item \texttt{stat}:显示文件信息
		\item \texttt{wc}:统计行数等信息
		\item \texttt{sort}:排序
		\item \sout{\texttt{tsort}:拓扑排序}
		\item \texttt{uniq}:去重
		\item \texttt{grep}:搜索文件内容
		\item \texttt{awk} / \texttt{cut} / \texttt{sed}:简单文件处理
		\item \texttt{head} / \texttt{tail}:截取文件 (\texttt{tail -f} 比较有用,通常用于输出日志)
		\item \texttt{du}:显示已用空间
		\item \texttt{df}:显示可用空间
		\item \texttt{less}:友好地展示文件内容 (允许手动滚屏)
		\item \texttt{realpath}:显示经过解析的绝对路径 (所有符号链接均经过解析,从根目录开始)
		\item \texttt{basename} / \texttt{dirname}:分离目录名和文件名
	\end{itemize}

	\subsubsection[网络]{网络\protect\cite{lbwvssl}}

	\begin{itemize}
		\itemsep0pt
		\item \texttt{ifconfig} / \texttt{ip}:查询网络配置
		\item \texttt{netstat}:查询连接情况
		\item \texttt{ping}:检查网络是否连通
		\item \texttt{tcpdump}:抓网络包
		\item \texttt{nc}:手动发/听网络包
		\item \texttt{dig} / \texttt{nslookup}:DNS 解析
		\item \texttt{curl} / \texttt{wget}:网络下载
	\end{itemize}

	\subsubsection{其它}

	\begin{itemize}
		\itemsep0pt
		\item \texttt{clear} 清屏
		\item \texttt{exec} 用接下来的程序替换当前进程
		\item \texttt{history} 查询历史记录
		\item \texttt{tar} / \texttt{gzip} / \texttt{gunzip} / \texttt{zip} / \texttt{unzip}:各种不同格式的压缩工具
		\item \texttt{shutdown} / \texttt{halt} / \texttt{reboot}:关机及重启
		\item \texttt{date}:显示或更改系统时间
		\item \texttt{uname}:查询系统相关信息 (\texttt{-a})
		\item \texttt{lsb\_release}:(Ubuntu) 查询版本信息 (\texttt{-a})
		\item \texttt{service} / \texttt{systemctl} 操作系统服务
		\item \texttt w / \texttt{who}:查询登录用户
		\item \sout{\texttt{write}:聊天 (记住 Linux 是多用户操作系统) (可以在上某些课时恶搞让别人终端突然收到消息)}
		\item \sout{\texttt{wall}:广播消息}
		\item \sout{\texttt{factor}:分解质因数}
		\item \sout{\texttt{tty}:查询当前 tty}
	\end{itemize}

	\section{Git}

	\section{实战}

	\subsection{服务器连接}
	\label{sss:connserver}

	\textcolor{gray}{[\textit{前置技能:\texttt{ssh} 命令}]}

	我们会按照报名问卷的填写情况为大家分配服务器账号,在获取到你的服务器用户名、IP 地址与账号后,请使用以下命令连接到服务器:

	\begin{minted}[frame=single,linenos=true,rulecolor=blue]{shell}
ssh train@59.66.131.240 -p <port>
	\end{minted}

	所有人的初始账号均为 \texttt{train},初始密码均为 \texttt{sast2023}。具体的端口号 (即 \texttt{<port>} 一栏) 会通过 /* TODO */ 方式下发。

	\subsection{公钥登录 (推荐)}

	\textcolor{gray}{[\textit{前置技能:\texttt{ssh-keygen}, \texttt{ssh-copy-id} 命令}]}

	\textcolor{red}{在登录成功后请尽快配置公/私钥登录,并修改强密码或撤销密码登录,简要流程如下:}

	\begin{itemize}
		\itemsep0pt
		\item (在本地) 直接输入命令 \texttt{ssh-keygen} (注:可选参数 \texttt{-t ed25519},更多参数见命令手册),会在家目录的 \texttt{.ssh} 文件夹下生成一对公私钥 (\verb!id_rsa.pub! 和 \verb!id_rsa!)。
		\item (在本地) 运行 \texttt{ssh-copy-id train@59.66.131.240 -p <port>} (格式和 \ref{sss:connserver} 相同,更多参数见命令手册),输入密码后即可完成 ``免密登录'' 配置。
	\end{itemize}

	\subsubsection{撤销密码登录 (可选)}

	\textcolor{gray}{[\textit{前置技能:\texttt{sshd} 服务相关原理}]}

	当你获得管理员 (\texttt{root}/\texttt{sudo}) 权限后,修改 \verb!/etc/ssh/sshd_config! 文件:

	\begin{itemize}
		\itemsep0pt
		\item 将第 57 (约) 行的 \verb!#PasswordAuthentication yes! 改为 \texttt{PasswordAuthentication no};
		\item 运行 \texttt{\textcolor{gray}{sudo} services ssh \textcolor{red}{reload}} 重新读取配置。
	\end{itemize}

	\subsection{CTF --- 简介}

	为保证趣味性,我们沿用之前几届的传统,采用 CTF (\href{https://en.wikipedia.org/wiki/Capture_the_flag_(cybersecurity)}{Capture the Flag}) 的形式进行。你需要在服务器环境中寻找格式形如 \textcolor{fuchsia}{\texttt{sast2023\{*******\}}} 的字符串,其中 \texttt{*******} 可以包含任何非空白 ASCII 可见字符。我们将这种字符串称为 ``flag''。每当你找到一个 ``flag''。你就可以通过 \ref{sss:ctfsubmit} 中介绍的提交方式进行提交。\cite{sast2022-linux}

	\subsection{CTF --- 提交方式}
	\label{sss:ctfsubmit}

	\textcolor{gray}{[\textit{前置技能:简单的命令行网络工具,如 \texttt{curl}, \texttt{wget} 等}]}

	不提供额外的提交程序,请使用命令行发送网络请求的方式进行提交。推荐使用 \texttt{curl}。

	提交的命令如下 (尖括号 \textcolor{maroon}{\texttt< \texttt> \textbf 不}输入,花括号 \textcolor{maroon}{\texttt\{ \texttt\} \textbf 要}输入):

	\begin{minted}[frame=single,linenos=true,rulecolor=blue,escapeinside=||]{shell}
curl -d 'id=<你的学号>' -d 'flag=sast2023{*******}' http://59.66.131.240:50000/submit
	\end{minted}

	\vspace{-6pt}
	对于如外校等没有学号的用户,可以直接去掉学号一栏,即使用如下命令提交。此时该命令仅用于检测正确性,不提供累积计分与排名功能。

	\begin{minted}[frame=single,linenos=true,rulecolor=blue,escapeinside=||]{shell}
curl -d 'flag=sast2023{*******}' http://59.66.131.240:50000/submit
	\end{minted}

	\vspace{-6pt}
	输入该命令后,你的终端会返回你的提交是否有效。简单起见,如果返回以 \texttt{"Success":} 开头说明提交成功,否则会给出对应的错误信息。

	此外,你可以在浏览器直接访问 \url{http://59.66.131.240:50000/} 查看当前 ``flag'' 的获取情况以及排行榜。排行榜默认按照总分降序排列,同分按最后一次提交时间升序排列。你也可以点击表头更改排序方式。

	\subsection{CTF --- 注意事项与 FAQ}

	\begin{itemize}
		\itemsep0pt
		\item 所有的 ``flag'' 分为三大区:Linux, Git, Bonus,分别包含 9, 6, 6 个 ``flag'',占总数的 58, 36, 6 分。
		\item 藏匿的 ``flag'' 会以多种形式出现,\textit{一般而言},出现格式正确的字符串就是 ``flag'',除非有特殊说明,要求你寻找其中一个特定的 ``flag'' (这种现象在 Git 区中比较常见)。
		\item 欢迎与其他人交流讨论思路,以及向 ChatGPT/GPT-4 等工具求助。但请不要直接将 ``flag'' 告诉别人,这样会影响其他人的游戏体验。
		\item 提交 ``flag'' 的服务器和分配给你的服务器是同一个服务器,所以不要尝试做出伤敌一千自损八百的事情。
		\item 由于\raisebox{-3pt}{\includegraphics{assets/pigeon.pdf}\includegraphics{assets/pigeon.pdf}\includegraphics{assets/pigeon.pdf}}的原因没做复杂的身份验证,所以请只用自己的学号提交 \sout{(真有人喜欢给别人送分吗?)}。
		\item ``mails'' 文件夹用于获得服务器的管理员权限以进行进一步操作,该文件夹与 ``flag'' 无关,且解锁所有三个区的 ``flag'' 都无需用到该权限 (即使用 ``train'' 账号即可完成)。
		\item 如果你没有分配到服务器,你可以去 Docker Hub 中的 \url{https://hub.docker.com/r/jkjkmxmx/sast2023-linux-git} 下载原生的镜像来游玩,该镜像是 x86-64/amd64 架构的,最好保证你的计算机也是此架构 \sout{(好像用 arm64 的 QEMU 模拟会出现某些 flag 拿不到的问题)}。具体流程如下:
		\begin{minted}[frame=single,linenos=true,rulecolor=blue,breaklines=true]{shell}
docker pull jkjkmxmx/sast2023-linux-git:v2
docker run --privileged -d -p 10000:22 -p 10001:80 -p 10002:3306 -p 10003:10001 -p 10004:10002 -h sast2023 --name sast2023 jkjkmxmx/sast2023-linux-git:v2
ssh -p 10000 train@localhost
		\end{minted}
	\end{itemize}

	\subsection{CTF --- Linux 区}

	该区在 ``train'' 的家目录 (即 \texttt{/home/train}) 下的 ``puzzles'' 目录中进行。在这个目录中的每一个子目录 (如 ``envir'') 就代表一个谜题,\textbf{一个谜题恰对应一个 ``flag''}。

	每个谜题下一定存在一个\textit{与目录名同名}的可执行文件 (可以有多种形式:二进制文件,Shell 脚本等;以 ``envir'' 为例,即为 \texttt{envir/envir}),运行该文件并按照它的引导一步步获取 ``flag''。

	\subsection{CTF --- Git 区}

	该区在 ``train'' 的家目录下的 ``git'' 目录中进行。本区包含 6 个 flag,简要介绍如下:

	\subsubsection{Branches}

	\begin{itemize}
		\item 看看这个仓库里有哪些分支?
	\end{itemize}

	\subsubsection{Message}

	\begin{itemize}
		\item 提交信息不止只有标题哦,还有正文!
	\end{itemize}

	\subsubsection{History}

	\begin{itemize}
		\item 数据被覆盖了!看看怎么跳回之前的版本?
	\end{itemize}

	\subsubsection{Reflog}

	\begin{itemize}
		\itemsep0pt
		\item 该谜题在 ``reflog'' 分支下进行。
		\item 观察提交信息:``add correct Taylor formula'',这说明之前有可能加入了错误的公式。
		\item 但是它不在版本树里,这说明很可能被 \texttt{git reset} 过。那该如何恢复这样的 ``误操作'' 呢?
	\end{itemize}

	\subsubsection{Werewolf}

	\begin{itemize}
		\itemsep0pt
		\item 该谜题在 ``werewolf'' 分支下进行。
		\item \textbf{\textcolor{red}{state 文件下看起来有很多 ``flag'',但是这些基本都是假的。}}
		\item \textcolor{blue}{``预言家'' 告诉你:最终版本的所有 ``flag'' 中,唯一一个由 ``狼人'' (werewolf) 提供且没有被后面平民 (villager) 覆盖的 ``flag'' 才是真正的 ``flag''。}
	\end{itemize}

	\subsubsection{Debug}

	\begin{itemize}
		\itemsep0pt
		\item 该谜题在 ``debug'' 分支下进行。
		\item ``cat'' 对 \texttt{debug.cpp} 文件修改了很多次,但这些文件产生运行时错误了!``cat'' 想知道,究竟是哪一次提交,使这个程序第一次产生运行时错误呢\footnote{保证存在这样的提交,使得在其之前 \texttt{debug.cpp} 正常运行,它及其它之后均产生运行时错误。}?
		\item \textcolor{gray}{[\textit{该怎么寻找这样的提交呢?二分查找?如何在版本树中进行二分查找?}]}
		\item \textbf{\textcolor{properpurple}{只有这个提交正文里的 ``flag'' 才是真正的 ``flag'' 哦!}}
	\end{itemize}

	\subsection{CTF --- Bonus 区}

	该区会在任何可能的地方进行,它们\textit{有可能出现在服务器中,也有可能出现在其他任何地方}。作为彩蛋,这里不给出相关的提示,希望大家能开心地寻找。

	\section{致谢}

	本次作业参考了 2021 与 2022 年计算机系暑培 Linux 讲义与作业\cite{sast2022-linux}。本次作业镜像使用计算机系科协提供的 @Zeus 服务器进行分发,感谢它在 2\#-308B 日日夜夜勤勤恳恳地工作。本次作业的镜像上传到了 \href{https://hub.docker.com/r/jkjkmxmx/sast2023-linux-git}{Docker Hub},所编写代码将会在作业结束后全部开源,供后续培训设计时作为参考。

	\bibliographystyle{plain}
	\bibliography{handout}
\end{document}
