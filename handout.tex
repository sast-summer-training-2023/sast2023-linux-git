\documentclass{article}
\usepackage{extramarks, fancyhdr, minted}
\usepackage{main}

\pagestyle{fancy}
\lhead{\nouppercase{\lastleftmark}}
\rhead{\nouppercase{\lastrightmark}}

\def\refname{References}

\title{SAST\,2023 Summer Training\\CTF | A Warmup for Linux \& Git}
\author{jkjkmxmx @ yhx-12243}
\setdate{2023}{7}{9}

\begin{document}
	\maketitle
	\let\labelitemi\labelitemiii

	\section{Linux}

	\subsection{Introduction}

	Linux 是一种自由和开放源码的类 UNIX 操作系统,其内核由 Linus Torvalds 在 1991 年发布。

	Linux 也是自由软件和开放源代码软件发展中最著名的例子,只要遵循 GNU,任何个人和机构都可以自由地使用 Linux 的所有底层源代码,这使得它得到来自全世界软件爱好者和组织的开发支持。

	Linux 以各种形式被广泛应用在各个领域,包括但不限于:
	\begin{itemize}
		\itemsep0pt
		\item 服务器、主机、超级计算机;
		\item 嵌入式系统(机顶盒、移动设备等);
		\item 基础设施(红绿灯、工业传感器)。\cite{net9-linux}
	\end{itemize}

	\subsection{课前准备}

	参考链接:
	\begin{itemize}
		\itemsep0pt
		\item \url{https://learn.microsoft.com/zh-cn/windows/wsl/install}
		\item \url{https://blog.csdn.net/yushuzhen2008/article/details/104944579}
	\end{itemize}


	\subsection{man 和 tldr}

	\subsection{ssh}


	\section{Git}

	\section{实战}

	\subsection{服务器连接}
	\label{sss:connserver}

	\textcolor{gray}{[\textit{前置技能:\texttt{ssh} 命令}]}

	我们会按照报名问卷的填写情况为大家分配服务器账号,在获取到你的服务器用户名、IP 地址与账号后,请使用以下命令连接到服务器:

	\begin{minted}[frame=single,linenos=true,rulecolor=blue]{shell}
ssh train@59.66.131.240 -p <port>
	\end{minted}

	所有人的初始账号均为 \texttt{train},初始密码均为 \texttt{sast2023}。具体的端口号 (即 \texttt{<port>} 一栏) 会通过 /* TODO */ 方式下发。

	\subsection{公钥登录 (推荐)}

	\textcolor{gray}{[\textit{前置技能:\texttt{ssh-keygen}, \texttt{ssh-copy-id} 命令}]}

	\textcolor{red}{在登录成功后请尽快配置公/私钥登录,并修改强密码或撤销密码登录,简要流程如下:}

	\begin{itemize}
		\itemsep0pt
		\item (在本地) 直接输入命令 \texttt{ssh-keygen} (注:可选参数 \texttt{-t ed25519},更多参数见命令手册),会在家目录的 \texttt{.ssh} 文件夹下生成一对公私钥 (\verb!id_rsa.pub! 和 \verb!id_rsa!)。
		\item (在本地) 运行 \texttt{ssh-copy-id train@59.66.131.240 -p <port>} (格式和 \ref{sss:connserver} 相同,更多参数见命令手册),输入密码后即可完成 ``免密登录'' 配置。
	\end{itemize}

	\subsubsection{撤销密码登录 (可选)}

	\textcolor{gray}{[\textit{前置技能:\texttt{sshd} 服务相关原理}]}

	当你获得管理员 (\texttt{root}/\texttt{sudo}) 权限后,修改 \verb!/etc/ssh/sshd_config! 文件:

	\begin{itemize}
		\itemsep0pt
		\item 将第 57 (约) 行的 \verb!#PasswordAuthentication yes! 改为 \texttt{PasswordAuthentication no};
		\item 运行 \texttt{\textcolor{gray}{sudo} services ssh \textcolor{red}{reload}} 重新读取配置。
	\end{itemize}

	\subsection{CTF --- 简介}

	为保证趣味性,我们沿用之前几届的传统,采用 CTF (\href{https://en.wikipedia.org/wiki/Capture_the_flag_(cybersecurity)}{Capture the Flag}) 的形式进行。你需要在服务器环境中寻找格式形如 \textcolor{fuchsia}{\texttt{sast2023\{*******\}}} 的字符串,其中 \texttt{*******} 可以包含任何非空白 ASCII 可见字符。我们将这种字符串称为 ``flag''。每当你找到一个 ``flag''。你就可以通过 \ref{sss:ctfsubmit} 中介绍的提交方式进行提交。\cite{sast2022-linux}

	\subsection{CTF --- 提交方式}
	\label{sss:ctfsubmit}

	\textcolor{gray}{[\textit{前置技能:简单的命令行网络工具,如 \texttt{curl}, \texttt{wget} 等}]}

	不提供额外的提交程序,请使用命令行发送网络请求的方式进行提交。推荐使用 \texttt{curl}。

	提交的命令如下 (尖括号 \textcolor{maroon}{\texttt< \texttt> \textbf 不}输入,花括号 \textcolor{maroon}{\texttt\{ \texttt\} \textbf 要}输入):

	\begin{minted}[frame=single,linenos=true,rulecolor=blue,escapeinside=||]{shell}
curl -d 'id=<你的学号>' -d 'flag=sast2023{*******}' http://59.66.131.240:50000/submit
	\end{minted}

	\vspace{-6pt}
	对于如外校等没有学号的用户,可以直接去掉学号一栏,即使用如下命令提交。此时该命令仅用于检测正确性,不提供累积计分与排名功能。

	\begin{minted}[frame=single,linenos=true,rulecolor=blue,escapeinside=||]{shell}
curl -d 'flag=sast2023{*******}' http://59.66.131.240:50000/submit
	\end{minted}

	输入该命令后,你的终端会返回你的提交是否有效。简单起见,如果返回以 \texttt{"Success":} 开头说明提交成功,否则会给出对应的错误信息。

	此外,你可以在浏览器直接访问 \url{http://59.66.131.240:50000/} 查看当前 ``flag'' 的获取情况以及排行榜。排行榜默认按照总分降序排列,同分按最后一次提交时间升序排列。你也可以点击表头更改排序方式。

	\subsection{CTF --- 注意事项与 FAQ}

	\begin{itemize}
		\itemsep0pt
		\item 所有的 ``flag'' 分为三大区:Linux, Git, Bonus,分别包含 9, 6, 6 个 ``flag'',占总数的 58, 36, 6 分。
		\item 藏匿的 ``flag'' 会以多种形式出现,\textit{一般而言},出现格式正确的字符串就是 ``flag'',除非有特殊说明,要求你寻找其中一个特定的 ``flag'' (这种现象在 Git 区中比较常见)。
		\item 欢迎与其他人交流讨论思路,以及向 ChatGPT/GPT-4 等工具求助。但请不要直接将 ``flag'' 告诉别人,这样会影响其他人的游戏体验。
		\item 提交 ``flag'' 的服务器和分配给你的服务器是同一个服务器,所以不要尝试做出伤敌一千自损八百的事情。
		\item 由于\raisebox{-3pt}{\includegraphics{assets/pigeon.pdf}\includegraphics{assets/pigeon.pdf}\includegraphics{assets/pigeon.pdf}}的原因没做复杂的身份验证,所以请只用自己的学号提交 \sout{(真有人喜欢给别人送分吗?)}。
		\item ``mails'' 文件夹用于获得服务器的管理员权限以进行进一步操作,该文件夹与 ``flag'' 无关,且解锁所有三个区的 ``flag'' 都无需用到该权限 (即使用 ``train'' 账号即可完成)。
		\item 如果你没有分配到服务器,你可以去 Docker Hub 中的 \url{https://hub.docker.com/r/jkjkmxmx/sast2023-linux-git} 下载原生的镜像来游玩,该镜像是 x86-64/amd64 架构的,最好保证你的计算机也是此架构 \sout{(好像用 arm64 的 QEMU 模拟会出现某些 flag 拿不到的问题)}。具体流程如下:
		\begin{minted}[frame=single,linenos=true,rulecolor=blue,breaklines=true]{shell}
docker pull jkjkmxmx/sast2023-linux-git:v1
docker run --privileged -d -p 10000:22 -p 10001:80 -p 10002:3306 -p 10003:10001 -p 10004:10002 -h sast2023 --name sast2023 jkjkmxmx/sast2023-linux-git:v1
ssh -p 10000 train@localhost
		\end{minted}
	\end{itemize}

	\subsection{CTF --- Linux 区}

	该区在 ``train'' 的家目录 (即 \texttt{/home/train}) 下的 ``puzzles'' 目录中进行。在这个目录中的每一个子目录 (如 ``envir'') 就代表一个谜题,\textbf{一个谜题恰对应一个 ``flag''}。

	每个谜题下一定存在一个\textit{与目录名同名}的可执行文件 (可以有多种形式:二进制文件,Shell 脚本等;以 ``envir'' 为例,即为 \texttt{envir/envir}),运行该文件并按照它的引导一步步获取 ``flag''。

	\subsection{CTF --- Git 区}

	该区在 ``train'' 的家目录下的 ``git'' 目录中进行。本区包含 6 个 flag,简要介绍如下:

	\subsubsection{Branches}

	\begin{itemize}
		\item 看看这个仓库里有哪些分支?
	\end{itemize}

	\subsubsection{Message}

	\begin{itemize}
		\item 提交信息不止只有标题哦,还有正文!
	\end{itemize}

	\subsubsection{History}

	\begin{itemize}
		\item 数据被覆盖了!看看怎么跳回之前的版本?
	\end{itemize}

	\subsubsection{Reflog}

	\begin{itemize}
		\itemsep0pt
		\item 该谜题在 ``reflog'' 分支下进行。
		\item 观察提交信息:``add correct Taylor formula'',这说明之前有可能加入了错误的公式。
		\item 但是它不在版本树里,这说明很可能被 \texttt{git reset} 过。那该如何恢复这样的 ``误操作'' 呢?
	\end{itemize}

	\subsubsection{Werewolf}

	\begin{itemize}
		\itemsep0pt
		\item 该谜题在 ``werewolf'' 分支下进行。
		\item \textbf{\textcolor{red}{state 文件下看起来有很多 ``flag'',但是这些基本都是假的。}}
		\item \textcolor{blue}{``预言家'' 告诉你:最终版本的所有 ``flag'' 中,唯一一个由 ``狼人'' (werewolf) 提供且没有被后面平民 (villager) 覆盖的 ``flag'' 才是真正的 ``flag''。}
	\end{itemize}

	\subsubsection{Debug}

	\begin{itemize}
		\itemsep0pt
		\item 该谜题在 ``debug'' 分支下进行。
		\item ``cat'' 对 \texttt{debug.cpp} 文件修改了很多次,但这些文件产生运行时错误了!``cat'' 想知道,究竟是哪一次提交,使这个程序第一次产生运行时错误呢\footnote{保证存在这样的提交,使得在其之前 \texttt{debug.cpp} 正常运行,它及其它之后均产生运行时错误}?
		\item \textcolor{gray}{[\textit{该怎么寻找这样的提交呢?二分查找?如何在版本树中进行二分查找?}]}
		\item \textbf{\textcolor{properpurple}{只有这个提交正文里的 ``flag'' 才是真正的 ``flag'' 哦!}}
	\end{itemize}

	\subsection{CTF --- Bonus 区}

	该区会在任何可能的地方进行,它们\textit{有可能出现在服务器中,也有可能出现在其他任何地方}。作为彩蛋,这里不给出相关的提示,希望大家能开心地寻找。

	\section{致谢}

	本次作业参考了 2021 与 2022 年计算机系暑培 Linux 讲义与作业\cite{sast2022-linux}。本次作业镜像使用计算机系科协提供的 @Zeus 服务器进行分发,感谢它在 2\#-308B 日日夜夜勤勤恳恳地工作。本次作业的镜像上传到了 \href{https://hub.docker.com/r/jkjkmxmx/sast2023-linux-git}{Docker Hub},所编写代码将会在作业结束后全部开源,供后续培训设计时作为参考。

	\bibliographystyle{plain}
	\bibliography{handout}
\end{document}
